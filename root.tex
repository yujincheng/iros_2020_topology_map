%%%%%%%%%%%%%%%%%%%%%%%%%%%%%%%%%%%%%%%%%%%%%%%%%%%%%%%%%%%%%%%%%%%%%%%%%%%%%%%%
%2345678901234567890123456789012345678901234567890123456789012345678901234567890
%        1         2         3         4         5         6         7         8

\documentclass[letterpaper, 10 pt, conference]{ieeeconf}  % Comment this line out if you need a4paper

%\documentclass[a4paper, 10pt, conference]{ieeeconf}      % Use this line for a4 paper

\IEEEoverridecommandlockouts                              % This command is only needed if 
                                                          % you want to use the \thanks command

\overrideIEEEmargins                                      % Needed to meet printer requirements.

%In case you encounter the following error:
%Error 1010 The PDF file may be corrupt (unable to open PDF file) OR
%Error 1000 An error occurred while parsing a contents stream. Unable to analyze the PDF file.
%This is a known problem with pdfLaTeX conversion filter. The file cannot be opened with acrobat reader
%Please use one of the alternatives below to circumvent this error by uncommenting one or the other
%\pdfobjcompresslevel=0
%\pdfminorversion=4

% See the \addtolength command later in the file to balance the column lengths
% on the last page of the document

\usepackage{epsfig}
\usepackage{float}
\usepackage{threeparttable}
\usepackage{subfigure}
\usepackage{url}

% The following packages can be found on http:\\www.ctan.org
%\usepackage{graphics} % for pdf, bitmapped graphics files
%\usepackage{epsfig} % for postscript graphics files
%\usepackage{mathptmx} % assumes new font selection scheme installed
%\usepackage{times} % assumes new font selection scheme installed
\usepackage{amsmath} % assumes amsmath package installed
\usepackage{amssymb}  % assumes amsmath package installed

\usepackage{bigstrut,multirow}
	
\usepackage[colorlinks=false, pdfborder={0 0 0}]{hyperref}
\usepackage{cleveref}
\crefname{figure}{fig}{figs}

\usepackage{microtype}


\title{\LARGE \bf
Data-efficient Multi-robot Map Building with Topology Map and CNN-based Scene Descriptors
}


\author{ Zhaoliang Zhang$^{1}$, Jincheng Yu$^{1}$, Zhilin Xu$^{1}$ and Yu Wang$^{1}$ % <-this % stops a space
% \thanks{*This work was not supported by any organization} % <-this % stops a space
\thanks{$^{1}$Electronic Engineering Department,
        Tsinghua University, Beijing, China
        {\tt\small yjc16@mails.tsinghua.edu.cn, yu-wang@tsinghua.edu.cn}}%
}


\begin{document}

\maketitle
\thispagestyle{empty}
\pagestyle{empty}


%%%%%%%%%%%%%%%%%%%%%%%%%%%%%%%%%%%%%%%%%%%%%%%%%%%%%%%%%%%%%%%%%%%%%%%%%%%%%%%%
\begin{abstract}
Multi-robot map building is the basic task for many multi-robot applications.
\end{abstract}

\section{Introduction}
\label{sec:intro}

Multi-robot map building of unknown environments is a fundamental problem for multi-robot autonomous robotics, such navigation  \cite{tanner2005towards} and rescue \cite{baxter2007multi}.

Different from single-robot's huge on-board communication bandwidth, the limited communication resources between robots becomes the bottleneck of the multi-robot map building.
In order to do multi-robot map building in communication constrained environments, this paper proposes a data-efficient map representation and merging method based on topology map, called PRET.
% (Place Representor Embedded Topology map)
PRET uses CNN to generate place representor of each new comming place and embeds the representor to the topology map.
Based on the representors, the topology maps of different robots can be easily shared and merged with little communication resources.
Compared with the tranditional grid map. The data transfer is reduced by ?\%. The navigation path is only ?\% higher than the grid map. %结果稍微丰富下

\bibliographystyle{IEEEtran}
\bibliography{src/fpgaslam}

\end{document}
